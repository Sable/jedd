\documentclass{article}
\usepackage{hyperref}

\title{A brief user's guide to Jedd}
\author{Ond\v{r}ej Lhot\'ak}

\begin{document}
\maketitle

\newcommand{\jeddpath}[1]{\href{https://svn.sable.mcgill.ca/soot/jedd/trunk/#1}{\nolinkurl{#1}}}

\section{Preliminaries}
This mini-tutorial assumes that the reader has
read the paper~\cite{Lhotak:2004:JBR} about Jedd that
was presented at PLDI 2004, and is available at
\url{http://www.sable.mcgill.ca/publications/papers/#pldi2004},
as well as Chapter 3 and Appendix B of Ond\v{r}ej Lhot\'ak's
PhD thesis~\cite{Lhotak:2005:PAU}, available at \url{http://www.sable.mcgill.ca/~olhota/pubs/thesis-olhotak-phd.ps}.

\section{Example}
The Jedd distribution contains a directory called \jeddpath{examples}
containing sample Jedd code. Currently, it contains a single example,
\jeddpath{examples/pointsto}. This is a Jedd version of the BDD-based points-to
analysis from~\cite{Berndl:2003:PAU}.

\section{Jedd source files}
Source files to be processed by Jedd must have one of the extensions
{\tt .jedd} or {\tt .java}. It is customary to use the extension {\tt .jedd}
for files containing Jedd-specific constructs, and {\tt .java} for files
containing plain Java.

Jedd files should import the package {\tt jedd.*} from the Jedd runtime
library. This package contains interface classes with methods that can
be called by Jedd programs. In particular, the {\tt jedd.Jedd} class
is a singleton containing methods affecting the behaviour of Jedd in
general, and {\tt jedd.Relation} is an interface listing the methods
that can be called on any Jedd relation type. Jedd files should not
import the package {\tt jedd.internal.*}.

\section{Defining numberers, domains, attributes and physical domains}
The first step in writing a Jedd program is to define the numberers,
domains, attributes, and physical domains that will be used. A numberer
is a class that generates and maintains a mapping between objects and
non-negative integers, and must be implemented by the programmer. A
domain is a set of objects that will form the basis of Jedd relations.
Each domain must have an associated numberer for its objects. An
attribute is a domain with an assigned name, used to distinguish
multiple instances of a domain within the same relation. A physical
domain is a set of BDD bit positions to which Jedd may map an attribute
of a relation.

A numberer is a plain Java class implementing the {\tt jedd.Numberer}
interface. See the file \jeddpath{examples/pointsto/src/domains/IntegerNumberer.java},
which assigns to {\tt Integer} objects the integer that is their value.

A domain is similar to a Java class, but is defined with a
slightly different syntax: the class name is immediately followed
by an integer constant in parentheses. See the file 
\jeddpath{examples/pointsto/src/domains/Var.jedd} for an example.
The integer constant specifies how many bits are to be used to represent
the domain. The maximum number of objects in the domain is $2^b$, where $b$
is the number of bits specified. Each domain must extend the {\tt jedd.Domain}
class and implement the {\tt numberer()} method, which returns the
numberer for the domain.

An attribute is defined similarly to a domain, but the integer constant
number of bits is replaced with the name of the domain of the attribute.
See the file \jeddpath{examples/pointsto/src/attributes/var.jedd} for an
example with the domain {\tt Var}. An attribute must extend the 
{\tt jedd.Attribute} class. However, it should not implement its
abstract method {\tt domain()}; Jedd will implement it for you.

A physical domain is defined similarly to a domain or attribute, but
the parentheses following its name are empty. See the file 
\jeddpath{examples/pointsto/src/physical_domains/V1} for an example.
Each physical domain must extend the class {\tt jedd.PhysicalDomain}.

\section{Selecting a backend}
Jedd currently supports four different BDD libraries as backends:
BuDDy, CUDD, SableJBDD, and JavaBDD. BuDDy is the backend which has
the most complete support in Jedd, which is the most tested, and which
tends to perform best. BuDDy and CUDD are C libraries, so they require
that their shared library ({\tt .so} or {\tt .dll}) files be available
on the {\tt LD\_LIBRARY\_PATH}. Before using Jedd in you program,
you must select one of the backends by calling
{\tt jedd.Jedd.v().setBackend()}. The argument to this method
should be one of {\tt "buddy"}, {\tt "cudd"}, {\tt "sablejbdd"} or
{\tt "javabdd"}.

\section{Selecting a physical domain ordering (optional)}
By default, Jedd places the various physical domains one after the next
in the BDD. For performance reasons, you may want to select a different
ordering. This is done by calling {\tt jedd.Jedd.v().setOrder()}.
An example of how this method is called appears in the points-to
analysis example 
(see \jeddpath{examples/pointsto/src/Prop.jedd}). A detailed
explanation of the orderings that can be specified appears in Ond\v{r}ej
Lhot\'ak's Ph.D.\ thesis~\cite{Lhotak:2005:PAU}, in the section
titled ``Specifying physical domain ordering'' in Chapter~3.

\section{Writing Jedd code}
The Jedd grammar and explanations of its operators appear in~\cite{Lhotak:2004:JBR,Lhotak:2005:PAU},
and are outside the scope of this guide. The paper also includes various
examples of Jedd code. Refer also to the points-to analysis example in
\jeddpath{examples/pointsto/src/Prop.jedd}.

The javadoc documentation of the (rather small) API available to Jedd programs
is available in \jeddpath{doc/api}. In particular, this includes the
{\tt jedd.Jedd} class with methods controlling the behaviour of Jedd in general,
and the {\tt jedd.Relation} interface of methods that can be called on any
relation type.

\section{Compiling Jedd code}
The Jedd compiler is invoked with the command {\tt java jedd.Main}.
It uses the same command-line format as Polyglot, with two additional
switches for specifying the path to a SAT solver ({\tt -s}) and a
SAT core extractor ({\tt -sc}). The simplest way to compile a project
is to list all the {\tt .jedd} files on the command line. This will
compile them to {\tt .java} files, and run {\tt javac} on them to
compile them to classfiles. The {\tt -c} switch disables the {\tt javac}
pass. If your project consists of both {\tt .jedd} and {\tt .java}
files, you can put them all on the command line, but be warned that
Polyglot will overwrite your {\tt .java} files unless you specify
an alternate output directory with the {\tt -d} switch.

The points-to analysis example includes a simple Ant build file
which can be modified for use in other projects.

\section{Using the profiler (optional)}
To use the profiler, it must be enabled before the computation to be profiled
begins by calling {\tt jedd.Jedd.v().enableProfiling()}. At the end of
the computation, the recorded profiling data can be written to a file
in SQL format by calling {\tt jedd.Jedd.v().outputProfile()} with a
{\tt java.io.PrintStream}. See the file \jeddpath{examples/pointsto/src/Prop.jedd}
for an example use of the profiler.

Viewing the profile data requires an SQL database and a CGI-capable web
server. The CGI scripts (found in the \jeddpath{profile_view} directory in the Jedd
distribution) are specific to SQLite, but should work with any
web server. They expect the profiling data in a database called
{\tt profile.db}, in the same directory as the scripts. This file can be
generated by piping the SQL file to SQLite with the command\\
{\tt cat profile.sql | sqlite profile.db}\\
(assuming the SQL file
is {\tt profile.sql}). thttpd can be started with the command:\\
{\tt /usr/sbin/thttpd -d /directory/with/cgi/scripts -p 8080 -c '*.cgi'}
(where {\tt /directory/with/cgi/scripts} is replaced with the directory
containing the Jedd CGI scripts from \jeddpath{profile_view}). This starts
the web server on port 8080. To view the profiling data, point your
web browser to\\
\url{http://127.0.0.1:8080/main.cgi}.

\bibliographystyle{alpha}
\bibliography{local}

\end{document}
